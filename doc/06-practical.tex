\section{Partie pratique}
\label{sec:practical}

\subsection{Implémentation de l'application}
L'implémentation de l'application a été réalisée en suivant une approche modulaire et structurée, en utilisant les meilleures pratiques de développement. Les principaux aspects de l'implémentation comprennent:

\subsubsection{Architecture du projet}
\begin{itemize}
    \item Structure MVC (Model-View-Controller)
    \item Organisation modulaire des fonctionnalités
    \item Séparation claire des responsabilités
    \item Gestion d'état avec Riverpod
\end{itemize}

\subsubsection{Fonctionnalités principales}
\begin{itemize}
    \item Système d'authentification sécurisé
    \item Génération et lecture de QR codes
    \item Suivi en temps réel des présences
    \item Interface adaptative (responsive design)
\end{itemize}

\subsubsection{Tests et assurance qualité}
\begin{itemize}
    \item Tests unitaires
    \item Tests d'intégration
    \item Tests de performance
    \item Revue de code systématique
\end{itemize}

\subsection{Caractéristiques des matériels utilisés}
Les caractéristiques matérielles minimales requises pour le bon fonctionnement de l'application sont:

\subsubsection{Configuration du serveur}
\begin{itemize}
    \item \textbf{Processeur:} Intel Xeon E5-2680 v4 ou équivalent
    \item \textbf{Mémoire RAM:} 16 GB DDR4
    \item \textbf{Stockage:} SSD 256 GB
    \item \textbf{Bande passante:} 100 Mbps minimum
\end{itemize}

\subsubsection{Configuration client (Desktop)}
\begin{itemize}
    \item \textbf{Processeur:} Intel Core i5 (8ème génération) ou équivalent
    \item \textbf{Mémoire RAM:} 8 GB minimum
    \item \textbf{Stockage:} 128 GB disponibles
    \item \textbf{Écran:} Résolution minimale de 1920x1080
\end{itemize}

\subsubsection{Configuration mobile}
\begin{itemize}
    \item \textbf{Processeur:} Snapdragon 665 ou équivalent
    \item \textbf{Mémoire RAM:} 4 GB minimum
    \item \textbf{Stockage:} 64 GB minimum
    \item \textbf{Caméra:} 8 MP minimum avec autofocus
\end{itemize}

\subsection{Caractéristiques des logiciels utilisés}
L'environnement logiciel nécessaire pour le développement et le déploiement de l'application comprend:

\subsubsection{Environnement de développement}
\begin{itemize}
    \item \textbf{Système d'exploitation:} Windows 10/11, macOS, Linux
    \item \textbf{IDE:} Visual Studio Code avec extensions Flutter/Dart
    \item \textbf{SDK:} Flutter 3.0 ou supérieur
    \item \textbf{Outils de versioning:} Git 2.30 ou supérieur
\end{itemize}

\subsubsection{Technologies utilisées}
\begin{itemize}
    \item \textbf{Framework:} Flutter
    \item \textbf{Language:} Dart 2.17 ou supérieur
    \item \textbf{Base de données:} PostgreSQL 13 ou supérieur
    \item \textbf{Backend:} Supabase
\end{itemize}

\subsubsection{Dépendances principales}
\begin{itemize}
    \item Flutter Riverpod pour la gestion d'état
    \item QR Flutter pour la génération de codes QR
    \item Mobile Scanner pour la lecture de codes QR
    \item Supabase Flutter SDK
\end{itemize}

\subsection{Présentation des interfaces}

\subsubsection{Interface desktop}
\begin{figure}[H]
    \centering
    \framebox{
        \begin{minipage}{0.8\textwidth}
            \centering
            \vspace{0.5cm}
            [Desktop Teacher Dashboard]\\
            \vspace{1cm}
            Screenshot montrant l'interface bureau avec:\\
            \vspace{0.3cm}
            - Navigation latérale\\
            - Liste des cours\\
            - Statistiques de présence\\
            - Génération de QR codes\\
            \vspace{0.5cm}
        \end{minipage}
    }
    \caption{Interface Bureau - Tableau de Bord Enseignant}
    \label{fig:desktop_teacher_dashboard}
\end{figure}

\subsubsection{Interface mobile}
\begin{figure}[H]
    \centering
    \framebox{
        \begin{minipage}{0.8\textwidth}
            \centering
            \vspace{0.5cm}
            [Mobile Teacher Interface]\\
            \vspace{1cm}
            Screenshot montrant l'interface mobile avec:\\
            \vspace{0.3cm}
            - Menu de navigation\\
            - Vue des présences\\
            - Contrôles QR code\\
            - Options rapides\\
            \vspace{0.5cm}
        \end{minipage}
    }
    \caption{Interface Mobile - Application Enseignant}
    \label{fig:mobile_teacher_interface}
\end{figure}